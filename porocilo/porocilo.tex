\documentclass[11pt,a4paper]{article}

\usepackage[slovene]{babel}
\usepackage[utf8x]{inputenc}
\usepackage{graphicx}
\usepackage{hyperref}
\usepackage{pdfpages}
\usepackage{breakurl}

\pagestyle{plain}

\begin{document}
\title{Poročilo pri predmetu \\
Analiza podatkov s programom R}
\author{Anja Skube}
\maketitle

\section{Izbira teme}

V projektu bom analizirala 50 najbolj prodajanih knjig vseh časov na svetu. Primerjala bom knjige po:

\begin{itemize}
\item jeziku izvirnika
\item letu izdaje
\item vrsti
\item spolu pisatelja
\item številu prodanih izvodov
\item ali so po knijgi posneli film
\item koliko del je pisatelj napisal
\end{itemize}

Začetno tabelo za moj projekt sem dobila na Wikipediji, ostale podatke pa sem morala poiskati na različnih straneh. Podatke sem nato uvozila v csv obliki in iz njih naredila eno tabelo. 

Povezava do podatkovne tabele:
\url{http://en.wikipedia.org/wiki/List_of_best-selling_books}

Cilj projekta je ugotoviti katero vrsto knjig ljudje najraje berejo, ali na prodajanost knjige vpliva spol pisatelja, kako število prodanih izvodov vpliva na to, ali je po knjigi posnet film ali ne ter iz katere države prihajajo pisatelji, ki so napisali 50 najbolj prodajanih knjig na svetu. 



\section{Obdelava, uvoz in čiščenje podatkov}
Uvozila sem 1 tabelo in ji dodala en manjkajoči stoplec(Država). Tabelo sem dobila iz Wikipedije in jo uvozila v CSV obliki.

Po uvozu tabele sem začela z risanjem grafov. Naredila sem tri grafe. 

Za prvi graf sem uporabila podatke o vrsti knjige. Naredila sem graf, ki primerja vrsto knjige (roman, novela, priročnik, pravljica...) in prikaže število posamezne vrste knjig. S tem sem želela pokazati, kakšen delež med 50 najbolj prodajanimi knjigami predstavla posamezna vrsta knjig.

V drugem grafu sem sem primerjala spol pisateljev, saj sem želela videti vpliv spola na prodajanost knjig. Ugotovila sem, da je prevladuje moški spol, to pa verjetno zaradi tega, ker veliko knjig sega se v 18. stoletje, takrat pa so pisali samo moški. 

V tretjem grafu pa sem uporabila podatke o filmih. Želela sem ugotoviti, kakšno je število knjig po katerih so posneli film. 

Za prvi in tretji graf sem izbrala stolpično opbliko (barplot), drugega pa sem naredila kot tortni diagram. Uvozila pa sem jih v pdf obliki. 

\includepdf{../slike/grafi.pdf}
\includepdf{../slike/tortni.pdf}
\includepdf{../slike/graf2.pdf}

\section{Analiza in vizualizacija podatkov}


V tretji fazi projekta sem narisala zemljevid sveta , ki prikazuje število knjig po državah.

% \makebox poravna element, ki je lahko širši od besedila
\makebox[\textwidth][c]{
\includegraphics[width=1.6\textwidth]{../slike/knjige.pdf}
} 



\section{Napredna analiza podatkov}


\end{document}
